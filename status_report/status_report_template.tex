    
\documentclass[11pt]{article}
\usepackage{times}
\usepackage{fullpage}
    
\title{The Sensitivity of Facial Analysis Algorithms to Race and Gender}
\author{Mohammed Zeerak - 2314940z}

\begin{document}
\maketitle


     

\section{Status report}

\subsection{Proposal}\label{proposal}

\subsubsection{Motivation}\label{motivation}

Facial recognition and computer vision have been employed in many instances throughout the 21st Century and with the increase in computational power, there has been a proportionate increase in the accuracy and speed of these facial detectors. Although this being said there is a lot of evidence to indicate that facial detection isn't getting more accurate and in fact its accuracy depends on factors such as race and gender. This inherent bias must be overcome if facial recognition is to be used successfully in a diverse setting.

\subsubsection{Aims}\label{aims}

This project will implement a few of the algorithms throughout the computer vision scene. The algorithms will range from the base foundational methods to more recent and complex deep CNN methods. Their performance against a data set comprising of represented and under represented faces will be calculated. These results will be analysed to understand if/why there exists a bias within certain algorithms. 

\subsection{Progress}\label{progress}

\begin{itemize}
\item Algorithms researched and chosen to satisfy representing the computer vision scene within the 21st Century.
\item Paper Summaries and Literature review completed on facial detection and the area of research.
\item  Viola and Jones algorithm implemented in Google Colab environment.
\item  DLIB algorithm implemented in Google Colab environment.
\item  MTCNN algorithm implemented in Google Colab environment.
\item  Retina Face algorithm implemented in Google Colab environment.
\item  Created a data set comprising of 48 photos of celebrities. 24 represented faces and 24 under represented faces with an even amount of males and females within the data set.
\item  Calculated ground truth bounding boxes and facial landmark locations through manual annotation of my data set.
\item  Executed each algorithm on my data set and aggregated the results within a evaluation notebook.
\item  Generated histograms showing the error difference between my ground truth the algorithms results for represented and underrepresented faces.
\item  Generated line graph showing the number of errors varying at each error threshold for the two sets of faces.
\item  Created a way to impose each of the algorithms results on the same face at the same time as to draw qualitative analysis from.
\item  Generated the faces which produced the maximum error and minimum error for each of the algorithms.
\item  Used subsets of my algorithms results to calculate a mean and standard deviation for my data set to evaluate its robustness
\item  Resized my data set so that every picture was of 300 width whilst maintaining the same proportions.
\item  Generated graphs looking at the intersection over union for the bounding boxes generated by the algorithms compared to the ground truth for both sets of faces.
\item  Calculated average error differences for each facial landmark for each algorithm across represented and under represented faces.
\item  Aggregated figures and metrics into a report and gave a short summary of each figure. 

\end{itemize}

\subsection{Problems and risks}\label{problems-and-risks}

\subsubsection{Problems}\label{problems}

\begin{itemize}
\item Large list of potential algorithms had to be narrowed down due to complexities implementing them on Google Colab environment.
\item Data set had to be resized to 300 width dimensions as it was returning inconsistent results across the different algorithms.
\item I had to reduce the data set from my planned 100 images to only 48 because of difficulties sourcing creative commons licensed photos.

\end{itemize}

\subsubsection{Risks}\label{risks}

\begin{itemize}
\item Many different routes to take research after my first set of results. \textbf{Mitigation}: will discuss with supervisor and decide next semester
\item Difficult to conclude why certain algorithms don't show a bias where as others do. \textbf{Mitigation}: will research more into the specific algorithms to understand why they don't introduce a bias like I thought they would
\item  Lack of information on data sets the different algorithms are trained on. \textbf{Mitigation}: Will look into reaching out to the authors of the papers to ask for information about the data-sets. Will also continue to further research into this
\end{itemize}


\subsection{Plan}\label{plan}

\begin{itemize}
    \item
      Week 1-3: Decide direction of analysis and look into specific faces. Start to write the dissertation and consider topics \textbf{Deliverable:}
      Results on specific facial features that affect the accuracy of the detectors.
    \item
      Week 4-6: Perform analysis on all my results and understand the reasons behind the bias if there exists one \textbf{Deliverable:} Report of results including analysis in comparison to algorithms
    \item 
      Week 6: Research into data sets that the algorithms are trained on and why this could be a reason for a bias to exist or not exist
    \item
      Week 7-10: Dissertation write up. Write the dissertation using my reports and analysis on the algorithms and data set. \textbf{Deliverable:} Submit first draft to supervisor
    \end{itemize}

    
\subsection{Ethics and data}\label{ethics}
This project does not involve human subjects or data. No approval required.
  


\end{document}
